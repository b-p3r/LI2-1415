\chapter*{Introdução} 
\addcontentsline{toc}{chapter}{Introdução} 

%Será discutido no capítulo %\ref{cap:devel} , na pagina %\pageref{cap:devel} os detalhes sobre o
%desenvolvimento da aplicação\\[1cm]

O projeto \emph{Haskassone} consiste numa adaptação do famoso jogo de tabuleiro \emph{Carcassone} no âmbito
do trabalho prático da cadeira de Laboratório de Informática I, do 1º ano do Curso da Licenciatura em
Engenharia Informática, do ano letivo de 2013/2014. Este projeto, para além de ser um meio de perceber o
processo de implementação de \emph{software} em todas as suas etapas, é uma forma de entender o tipo de
\emph{software}, em particular, de jogos de computador. Para compreender este \emph{software} específico é
necessário ter em mente três características essenciais ao software do género: \emph{interface}
gráfico, inteligência artificial e motor do jogo. Estas funcionalidades serão implementadas no programa
\texttt{Draw}, o programa \texttt{Play} e o programa \texttt{Next}, respetivamente.


Em geral, um jogo de computador necessita de três componentes fundamentais. O jogo necessita da componente
que cria um \emph{interface} de forma a tornar inteligível, o que se passa na máquina, bem como necessita da
inteligência artificial própria dos adversários num jogo de computador, como também necessita o motor do
jogo para reger as regras do universo do jogo. Além destas funcionalidades, podem existir outras, tais como,
modo multi-jogador, comunicações em rede, etc., no entanto, as três características mencionadas são as
fundamentais para qualquer software do género.

Em particular, o \emph{Haskassone} herda, naturalmente as três características fundamentais dos jogos de
computador. Assim, é necessário que o jogo tenha um \emph{interface} inteligível para o ser humano, neste
caso no formato de um tabuleiro com as suas peças ordenadas, necessita da componente da inteligência
artificial própria do adversários do utilizador, que produza jogadas válida seguindo uma estratégia, como
também é necessário o motor do jogo, para criar as regras do \emph{Haskassone}, tais como atribuições de
pontos, atualizações do estado do jogo, etc. O projeto incide, em grande parte, sobre estas três
características e para tal é necessário criar três programas distintos: um programa \texttt{Draw} que
será responsável pela criação de um \emph{interface} gráfico, um programa \texttt{Play} que levará a
cabo as tarefas da inteligência artificial característica dos adversários num jogo, e um programa
\texttt{Next} que se responsabiliza pela gestão do jogo.

Assim, o projeto terá por objetivo a implementação do \emph{Haskassone}, com os três programas, cujas as
funcionalidades são as três características fundamentais dos jogos de computador, através da análise de
cada problema, a estratégia, ou abordagem, de implementação de todas as funcionalidades necessárias aos
programas, bem como, que testes efetuados, os \emph{bugs} encontrados e a sua resolução e, possivelmente uma
breve discussão sobre a resolução do problema (formas alternativas de implementação, sugestões de
otimização, conselhos recebidos, etc.)






