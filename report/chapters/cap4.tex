\chapter{Programa Next}

\section{Análise do Problema}

%1. Hipótese / Objetivo. Em primeiro lugar, referir o que se está a tentar descobrir, ou tentar fazer, da
%froma mais clara possível

No programa Next, dada a descrição do tabuleiro em XML no stdin, tem o objetivo, para o caso de o jogo ainda
nao tenha acabado, ir buscar uma peça aleatoriamente e coloca-la no elemento next. Caso já se tenham jogadas
todas as  peças, o jogo  dá-se por terminado  e não é  gerado mais nenhum  elemento next. O  programa Next
também tem  o objetivo de calcular  as pontuações durante  o jogo e  retirar os meeples que  já pontuaram,
além de que o programa usa as regras de fim de jogo para recalcular as pontuações. O Next devolve no stdout
uma nova representação do tabuleiro em XML.
--processa :: Int -> Element -> String

--ler ficheiro XML
--ler jogada recebida

  --  2 pontos por tile de cidade, caso seja fechada (retira o meeple)

  {-- 9 pontos por claustro (mantém o meeple)
        cidade com n meeples (pontuação dividida por n jogadores)--}

--caso colocação da peça, verificar se fecha alguma coisa
--caso contrário não conta

--atualizar a pontuação do jogador que fecha algo

--escrever ficheiro (adicionar element a lista de elements (com ou sem meeple)

-- escrever no elemento scores (elemento player atributo player pontuação atual)

-- ler fichero
--caso fim de jogo contar todos 

-- com novo(s)  datatype (s) criar estrutura 

	-- cidade CLOSED | OPEN
	-- claustro CLOSED | OPEN
	

	-- Sequência de cidades 
        -- Sequência de claustros  
	-- Sequência de quintas


  {-- fim de jogo contam-se pontuam cidades incompletos que tenha um K
        1 ponto por cada tile que contenha o contenha--}

  {-- fim de jogo contam-se os claustros incompletos que contenham um M
    	1 pelo M e 1 por cada tile que o rodeia--}
  {-- fim de jogo pontos pelos F
    	-- contar 3 pelos F por cada cidades abastecida completa não interessa a distância
    	-- Vários F podem abastecer uma cidade 
		- Maior número de pontos para o que tiver mais a abastecer uma cidades.
		- Se empatados dividir pelo númerode jogadores--}




%2. A experiência anterior. Descrever informações relevantes a partir da leitura efetuada, conselhos
%recebidos,
%e quaisquer outros factos relevantes.

\section{Implementação e Testes}

%3.(Planeamento/Implementação/Testes). Referir como foi projetado o programa, as simulações, incluindo o que
%os
%resultados devem ser considerados para confirmar a hipótese, e que resultados deveriam esperados para
%rejeitá-la (com apoio de matemática,se for o caso).

%4. Execução. O que aconteceu quando se executou o programa, o que aconteceu de errado, que 
%coisas inesperadas aconteceram (se existirem).

%5. Os resultados. Foi a hipótese confirmada ou rejeitada?

\section{Conclusões}
%6. Conclusões. Incluindo o que teria sido feito de uma forma melhor.





